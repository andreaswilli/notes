\documentclass[../main.tex]{subfiles}

\firstpageheader{6.042}{Problem Set 2}{Page \thepage\ of \numpages}
\runningheader{6.042}{Problem Set 2}{Page \thepage\ of \numpages}

\begin{document}
\begin{questions}

\question We will show that any sequence of five distinct integers will contain a \textit{3-chain} (subsequence of three integers, which is monotonically increasing or decreasing (not necessarily contiguous)).
\begin{parts}
  \part Assume that $a_1 < a_2$. Show that if there is not \textit{3-chain} in our sequence, then $a_3$ must be less than $a_1$. (Hint: consider $a_4$!)
  \begin{solution}

    There are 3 possibilities to insert $a_3$:

    Case 1: $a_1 < a_2 < a_3$. In this case we immediately have a 3-chain.

    Case 2: $a_1 < a_3 < a_2$. Now, $a_4$ can be either greater or lower than $a_3$. If $a_4 < a_3$ we have the 3-chain $a_2 > a_3 > a_4$. Otherwise, we have the 3-chain $a_1 < a_3 < a_4$.

    Case 3: $a_3 < a_1 < a_2$. This is the only case where it's possible to have no 3-chain up to $a_4$. In either $a_3 < a_4 < a_1 < a_2$ or $a_3 < a_1 < a_4 < a_2$ there is no 3-chain.
  \end{solution}

  \part Using the previous part, show that if $a_1 < a_2$ and there is no 3-chain in our sequence, then $a_3 < a_4 < a_2$.
  \begin{solution}

    If $a_4 < a_3$, there is a 3-chain: $a_4 < a_3 < a_1$, since $a_3 < a_1$ by part (a).
    
    Similarly, if $a_2 < a_4$, there is a 3-chain: $a_1 < a_2 < a_4$.

    Therefore, $a_3 < a_4 < a_2$ (two possible sequences shown in part (a)).
  \end{solution}

  \part Assuming that $a_1 < a_2$ and $a_3 < a_4 < a_2$, show that any value of $a_5$ must result in a 3-chain.
  \begin{solution}

    $a_5$ can be either greater or lower than $a_4$. If $a_5 < a_4$ there is a 3-chain $a_5 < a_4 < a_2$. Otherwise, there is also a 3-chain $a_3 < a_4 < a_5$. Therefore, any value of $a_5$ results in a 3-chain.
  \end{solution}

  \part Using the previous parts, prove by contradiction that any sequence of five distinct integers must contain a 3-chain.
  \begin{solution}

    Parts (a) through (c) show that assuming $a_1 < a_2$ there is no possible sequence that does not contain a 3-chain. By symmetry, the same holds for $a_2 < a_1$. Therefore, any sequence of five distinct integers must contain a 3-chain. $\blacksquare$
  \end{solution}
\end{parts}

\question Prove by either the Well Ordering Principle or Induction that for all nonnegative integers, $n$:

$$
\sum_{i=0}^{n} i^3 = \left(\frac{n(n+1)}{2}\right)^2
$$

\begin{solution}

  Proof by induction. Let the induction hypothesis $P(n)$ for $n >= 0$ be that the above statement holds for $n$.

  Base case: $P(0) = \left(\frac{0(0+1)}{2}\right)^2 = 0 = \sum_{i=0}^{0} i^3$. Thus, the base case holds.
  
  Inductive step: We assume $P(n)$ and show that $P(n+1)$ follows. We add $(n+1)^3$ to the right side of $P(n)$ and expand the terms:
  $$
  \left(\frac{n(n+1)}{2}\right)^2 + (n+1)^3
  $$
  $$
  = \frac{(n(n+1))^2 + 2^2(n+1)^3}{2^2}
  $$
  $$
  = \frac{n^2(n^2+2n+1) + 4(n^3+3n^2+3n+1)}{2^2}
  $$
  $$
  = \frac{n^4+2n^3+n^2+4n^3+12n^2+12n+4}{2^2}
  $$
  $$
  = \frac{n^4+6n^3+13n^2+12n+4}{2^2}
  $$

  Now, we expand the terms of $P(n+1)$:
  $$
  \left(\frac{(n+1)(n+2)}{2}\right)^2
  $$
  $$
  = \frac{(n+1)^2(n+2)^2}{2^2}
  $$
  $$
  = \frac{(n^2+2n+1)(n^2+4n+4)}{2^2}
  $$
  $$
  = \frac{n^4+4n^3+4n^2+2n^3+8n^2+8n+n^2+4n+4}{2^2}
  $$
  $$
  = \frac{n^4+6n^3+13n^2+12n+4}{2^2}
  $$

  Thus, we showed that:
  $$
  \sum_{i=0}^{n+1} i^3 = \left(\frac{n(n+1)}{2}\right)^2 + (n+1)^3 = \left(\frac{(n+1)(n+2)}{2}\right)^2
  $$
  Which means that $P(n) \implies P(n+1)$.

  We conclude by induction that $P(n)$ holds for all $n >= 0$. $\blacksquare$
\end{solution}

\end{questions}
\end{document}
