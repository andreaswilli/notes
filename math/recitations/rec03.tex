\documentclass[../main.tex]{subfiles}

\firstpageheader{6.042}{Recitation Problems 3}{Page \thepage\ of \numpages}
\runningheader{6.042}{Recitation Problems 3}{Page \thepage\ of \numpages}

\begin{document}
\begin{questions}

  \question Breaking a chocolate bar
  \begin{solution}
    Proof. By strong induction. Let the induction hypothesis $P(n)$ for $n \ge 1$ be that any chocolate bar with area $n$ can be broken into its $n$ individual squares in $n-1$ splits.

    Base case: $P(1)$ is true, since it is already a single $1 \times 1$ square, thus 0 splits are needed.

    Inductive step: Assume $P(m)$ for $1 \le m \le n$ to show $P(n+1)$ follows. The chocolate bar with area $n+1$ can be split once into pieces of area $\frac{n+1}{w}*k$ and $\frac{n+1}{w}*(w-k)$, where $w$ is the width of the bar and $k$ is the position of the split. So by the induction hypothesis, the total amount of splits is:

    $$
    \frac{n+1}{w}*k-1 + \frac{n+1}{w}*(w-k)-1 + 1
    $$
    $$
    = \frac{nk+k+nw-nk+w-k}{w} - 1
    $$
    $$
    = \frac{nw+w}{w} - 1
    $$
    $$
    = n
    $$

    This shows that $P(n+1)$ holds. By strong induction we conclude, that $P(n)$ holds for every $n \ge 1$. We can also see that the number of splits does not depend on the position of the split $k$ (and neither on the shape of the bar, $w$). $\blacksquare$
  \end{solution}

  \question The Temple of Forever
  \begin{solution}

    State: $(r, g)$

    Transitions:
    \begin{itemize}
      \item Exchange: $(r, g) \rightarrow (r-3,\ g+2)$
      \item Swap: $(r, g) \rightarrow (g, r)$
    \end{itemize}

    \textbf{Theorem 1.} The state $(5, 5)$ cannot be reached from $(15, 12)$ by \textit{exchanging} and \textit{swapping}.

    Proof. We use induction on the number of gong rings. Let $P(n)$ be the proposition that after $n$ gong rings, the number of red beads $r$ minus the number of green beads $g$ is equal to $5k + 2$ or $5k + 3$ for some integer $k$.

    Base case: $P(0)$ is true, because for $(15, 12)$ the difference is $15-12 = 3 = 5*0+3$.

    Inductive step: Assume $P(n)$ to show that $P(n+1)$ follows. We look each transition case by case:

    Case 1 - exchange: The exchange transition decreases $r-g$ by $5$. Thus $r-g=5k+2$ or $r-g=5k+3$ will still be true after the transition.

    Case 2 - swap: The swap transition inverts the sign of $r-g$. We look at two subcases:

    Case 2.1 - $r-g=5k+2$: After the swap the sign will be inverted:

    $$
    g-r = -5k-2
    $$
    $$
    = -5k-5-2+5
    $$
    $$
    = 5(-k-1)+3
    $$

    If $k$ is an integer, $-k-1$ is still an integer and the proposition still holds.

    Case 2.2 - $r-g=5k+3$: Similarly:

    $$
    g-r = -5k-3
    $$
    $$
    = -5k-5-3+5
    $$
    $$
    = 5(-k-1)+2
    $$

    We can use the same argument as for case 2.1 to conclude that the proposition still holds.

    By induction we conclude, that P(n) holds for any $n \ge 0$. However, it does not hold for the state $(5, 5)$, because $5k = -2$ and $5k = -3$ do not have integer solutions. Thus, $(5, 5)$ cannot be reached by \textit{exchanging} and \textit{swapping}. $\blacksquare$

    \textbf{Theorem 2.} There is a finite number of reachable states in the Temple of Forever machine.

    Proof. We use induction on the number of gong rings. Let $P(n)$ be the proposition that after $n$ rings, the sum of beads $r+g$ is at most 27.

    Base case: P(0) holds because $15+12 = 27$.

    Inductive step: Assume $P(n)$ to show $P(n+1)$ follows. We look at the two cases:

    Case 1 - exchange:

    $$
    r-3+g+2 = r-g-1 < r-g \le 27
    $$

    The exchange reduces the sum by 1, thus it is still less than 27.

    Case 2 - swap: The swap does not change the sum, thus it is still less than 27.

    By induction $P(n)$ holds for any $n \ge 0$. This puts an upper bound on the number of reachable states, which means there is a finite number of reachable states. $\blacksquare$

    \textbf{Theorem 4.} It is not possible to visit 108 unique states in the Temple of Forever machine.

    Proof. The proof is by contradiction. Assume 108 unique states are reachable from $(15, 12)$ by transitioning using \textit{exchange} and \textit{swap}, since these are the only possible transitions.

    First we look at exchanges: In order to exchange, we need to have at least 3 red beads. At the same time $r-3+g+2 = r+g-1$ which means every exchange decreases the sum of beads by 1. We start with 27 beads, so exchanges can be performed at most 25 times.

    Now, let's look at swaps: The number of swaps is unlimited. However, performing two swaps in a row brings us to a state we already visited.

    So with the exchange we can transition between at most 26 unique states ($2 \le (r+g) \le 27)$. We can invert each of these by applying swap, which gives us at most 52 unique states. This contradicts our assumption, thus it is not possible to visit 108 unique states. $\blacksquare$
  \end{solution}

\end{questions}
\end{document}
