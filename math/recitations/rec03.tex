\documentclass[../main.tex]{subfiles}

\firstpageheader{6.042}{Recitation Problems 3}{Page \thepage\ of \numpages}
\runningheader{6.042}{Recitation Problems 3}{Page \thepage\ of \numpages}

\begin{document}
\begin{questions}

  \question Breaking a chocolate bar
  \begin{solution}
    Proof. By strong induction. Let the induction hypothesis $P(n)$ for $n \ge 1$ be that any chocolate bar with area $n$ can be broken into its $n$ individual squares in $n-1$ splits.

    Base case: $P(1)$ is true, since it is already a single $1 \times 1$ square, thus 0 splits are needed.

    Inductive step: Assume $P(m)$ for $1 \le m \le n$ to show $P(n+1)$ follows. The chocolate bar with area $n+1$ can be split once into pieces of area $\frac{n+1}{w}*k$ and $\frac{n+1}{w}*(w-k)$, where $w$ is the width of the bar and $k$ is the position of the split. So by the induction hypothesis, the total amount of splits is:

    $$
    \frac{n+1}{w}*k-1 + \frac{n+1}{w}*(w-k)-1 + 1
    $$
    $$
    = \frac{nk+k+nw-nk+w-k}{w} - 1
    $$
    $$
    = \frac{nw+w}{w} - 1
    $$
    $$
    = n
    $$

    This shows that $P(n+1)$ holds. By strong induction we conclude, that $P(n)$ holds for every $n \ge 1$. We can also see that the number of splits does not depend on the position of the split $k$ (and neither on the shape of the bar, $w$). $\blacksquare$
  \end{solution}

  \question The Temple of Forever

\end{questions}
\end{document}
