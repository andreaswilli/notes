\documentclass[../main.tex]{subfiles}

\firstpageheader{6.042}{Recitation Problems 4}{Page \thepage\ of \numpages}
\runningheader{6.042}{Recitation Problems 4}{Page \thepage\ of \numpages}

\begin{document}
\begin{questions}

  \question The Pulverizer
  \begin{parts}
    \part Describe a situation where the frog can't reach the worm.
    \begin{solution}
      The frog can't reach the worm if $gcd(k, n) > 1$. One example is $n = 10$ and $k = 6$.
    \end{solution}

    \part In a situation where the frog can actually reach the worm, explain how to use the Pulverizer to find how many jumps the frog will need.
    \begin{solution}
      Assuming the frog can reach the work, there is a number of jumps $j$ and a number of cycles $c$ so that $jk = cn + 1$. Thus $1$ is a linear combination of $k$ and $n$. Since $1$ is the smallest positive integer $jk - cn = 1$ needs to be the smallest positive linear combination of $k$ and $n$. This means that $1$ is the GCD of $k$ and $n$ and we can now use the Pulverizer to calculate $j$. 
    \end{solution}

  \part Compute the number of jumps if $n = 50$ and $k = 21$. Anything strange? Can you fix it?
  \begin{solution}
    In this case $j$ is equal to $-19$ (19 jumps in the other direction would be the shortest way to the worm). The Pulverizer is not guaranteed to give us a positive value for $j$, thus we need to find a positive $j$ some other way.
  \end{solution}
  \end{parts}

  \question Give an inductive proof that the Fibonacci numbers $F_n$ and $F_{n+1}$ are relatively prime for all $n \ge 0$.
  \begin{solution}

    Proof by Induction on $n$. Let $P(n)$ be the proposition that $F_n$ and $F_{n+1}$ are relatively prime.

    Base case: $P(0)$ is true, since $F_0 = 0$ and $F_1 = 1$ have no common divisors greater than $1$.

    Inductive step: We show that for all $n \ge 0$, $P(n) \implies P(n+1)$. In order to show this, we assume $P(n)$ and show that $P(n+1)$ follows. We will do this by contradiction.

    Suppose $F_{n+1}$ and $F_{n+2}$ are not relatively prime, which means they have a common divisor $d$ such that $\frac{F_{n+1}}{d}$ and $\frac{F_{n+2}}{d}$ are integer. Now consider this equation:

    $$
    F_{n+2} - F_{n+1} = d \cdot \frac{F_{n+2}}{d} - d \cdot \frac{F_{n+1}}{d} = d \cdot \left(\frac{F_{n+2}}{d} - \frac{F_{n+1}}{d}\right)
    $$

    This implies that $d$ divides $F_{n+2} - F_{n+1} = F_n$ and $d$ is a divisor of both $F_n$ and $F_{n+1}$ which contradicts the induction hypothesis. Thus $F_{n+1}$ and $F_{n+2}$ must be relatively prime and $P(n+1)$ holds.


    We conclude by induction that $P(n)$ holds for all $n \ge 0$. $\blacksquare$
  \end{solution}

  \question Extra Problem: The power of 3.
  \begin{solution}
    \textit{skipped}
  \end{solution}

\end{questions}
\end{document}
