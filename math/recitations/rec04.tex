\documentclass[../main.tex]{subfiles}

\firstpageheader{6.042}{Recitation Problems 4}{Page \thepage\ of \numpages}
\runningheader{6.042}{Recitation Problems 4}{Page \thepage\ of \numpages}

\begin{document}
\begin{questions}

  \question The Pulverizer
  \begin{parts}
    \part Describe a situation where the frog can't reach the worm.
    \begin{solution}
      The frog can't reach the worm if $gcd(k, n) > 1$. One example is $n = 10$ and $k = 6$.
    \end{solution}

    \part In a situation where the frog can actually reach the worm, explain how to use the Pulverizer to find how many jumps the frog will need.
    \begin{solution}
      Assuming the frog can reach the work, there is a number of jumps $j$ and a number of cycles $c$ so that $jk = cn + 1$. Thus $1$ is a linear combination of $k$ and $n$. Since $1$ is the smallest positive integer $jk - cn = 1$ needs to be the smallest positive linear combination of $k$ and $n$. This means that $1$ is the GCD of $k$ and $n$ and we can now use the Pulverizer to calculate $j$. 
    \end{solution}

  \part Compute the number of jumps if $n = 50$ and $k = 21$. Anything strange? Can you fix it?
  \begin{solution}
    In this case $j$ is equal to $-19$ (19 jumps in the other direction would be the shortest way to the worm). The Pulverizer is not guaranteed to give us a positive value for $j$, thus we need to find a positive $j$ some other way.
  \end{solution}
  \end{parts}

\end{questions}
\end{document}
