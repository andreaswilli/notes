\documentclass[../main.tex]{subfiles}

\firstpageheader{6.042}{Recitation Problems 5}{Page \thepage\ of \numpages}
\runningheader{6.042}{Recitation Problems 5}{Page \thepage\ of \numpages}

\begin{document}
\begin{questions}

  \question RSA: Beforehand steps
  \begin{parts}
    \part Choose primes $p$ and $q$ in range $5-15$ (would be hundrets of digits in practice).
    \begin{solution}
      We choose $p=7$ and $q=13$.
    \end{solution}

    \part Calculate $n=pq$ which will be used to encrypt and decrypt messages.
    \begin{solution}
      $n=7 \cdot 13=91$
    \end{solution}

    \part Find an $e>1$ such that $gcd(e, (p-1)(q-1))=1$.
    \begin{solution}
      $(p-1)(q-1)=72$. If we pick $e=5$, then $gcd(5, 72)=1$.
    \end{solution}

    \part Find a $d$ such that $de \equiv 1\ (mod\ (p-1)(q-1))$.
    \begin{solution}
      $5d \equiv 1\ (mod\ 72)$. This holds for $d=29$.

      So we have our keys:

      Public key: (5, 91)

      Private key: (29, 91)
    \end{solution}
  \end{parts}

  \question RSA: Pick a message to encrypt.
  \begin{solution}
    We pick the message $6=$ "Someone on our team thinks someone on your team is kinda cute."
  \end{solution}

  \question RSA: Encrypt the message using the public key.
  \begin{solution}

    $m'=rem(m^e,\ n)$

    Public key: (5, 91), message: 6

    $m'=rem(6^5, 91)=41$
  \end{solution}

  \question RSA: Decrypt the message using the private key. 
  \begin{solution}
    $m=rem((m')^d,\ n)$

    Private key: (29, 91), cypher text: 41

    $m=rem(41^{29},\ 91)=6$ which was in fact the original message.
  \end{solution}

  \question RSA: Explain how you could read messages encrypted with RSA if you could quickly factor large numbers.
  \begin{solution}
    The public key consists of $(e,\ n)$. If we can factor $n$ into $pq$ quickly, we can use these factors and the Pulverizer or Euler's Theorem to calculate the private key and decrypt all messages encrypted with the corresponding public key.
  \end{solution}
\end{questions}
\end{document}
