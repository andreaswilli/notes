\documentclass[../main.tex]{subfiles}

\firstpageheader{6.042}{Recitation Problems 7}{Page \thepage\ of \numpages}
\runningheader{6.042}{Recitation Problems 7}{Page \thepage\ of \numpages}

\begin{document}
\begin{questions}

  \question A Protocol for College Admission
  \begin{solution}

    Each day:

    \begin{itemize}
      \item Morning: Each student visits their favorite university that is not
        crossed off on their list.

      \item Afternoon: University $u_i$ tells their $n_i$ favorite applicants
        "maybe, come back tomorrow" and rejects all the other applicants of today.

      \item Evening: Every student that got rejected today crosses this
        university off their list.
    \end{itemize}

    Termination condition: On the day every university $u_i$ has at most $n_i$
    applicants, we stop.
  \end{solution}

  \begin{parts}
    \part Show that the algorithm terminates after $NM+1$ days.
    \begin{solution}
      Proof by contradiction. Assume the algorithm does not terminate after
      $NM+1$ days. On any day the algorithm does not terminate, there is at
      least one university that has more applicants than available spots
      (termination condition). Thus at least one student gets rejected and at
      least one university gets crossed off a list. After $NM+1$ days at least
      $NM+1$ universities have been crossed off. However, this is a
      contradiction since there is a total of $NM$ universities on lists. So
      the algorithm has to terminate on day $NM+1$ or earlier. $\blacksquare$
    \end{solution}

    \part Show the following: If during some day a university $u_j$ has at
          least $n_j$ applicants, then when the algorithm terminates it accepts
          exactly $n_j$ students.
    \begin{solution}
      On the day that $u_j$ has at least $n_j$ applicants it keeps its $n_j$
      favorites around. These get only rejected if a preferred student applies
      on a later day. Each applicant that has not been rejected will still want
      to join $u_j$ on the next day. Thus, on the next day $u_j$ still has at
      least $n_j$ applicants and the number of applicants can never go below
      $n_j$ at this point. On the day the algorithm terminates, the university
      can have at most $n_j$ students. Thus, on termination it must have
      exactly $n_j$ applicants. $\blacksquare$
    \end{solution}

    \part Show that every student is assigned to one university.
    \begin{solution}
      Proof by contradiction. Assume that one student is not assigned to a
      university on the day the algorithm ends. This means that one university
      $u_j$ admits less than $n_j$ students. Thus $u_j$ had less than $n_j$
      applicants each day (by previous proof), which means they never rejected
      anyone. Since every university is on a students list, they must have
      applied to it on one day. But this means that the student would be
      assigned to $u_j$ because it was their first choice at this point.
      However, this student is not assigned to a university and we have a
      contradiction. $\blacksquare$
    \end{solution}

    \part Show that for all $i$, $u_i$ gets assigned $n_i$ students.
    \begin{solution}
      The total number of spots is equal to the total number of students. Each
      student is assigned to exactly one university. Thus for every $i$,
      university $u_i$ must have admitted exactly $n_i$ students. $\blacksquare$
    \end{solution}

    \part Suppose that on some day a university $u_j$ has at least $n_j$
          applicants. Show that the rank of $u_j$'s least favorite applicant
          that it says "Maybe..." to cannot decrease on any future day.
    \begin{solution}
      $u_j$ will be the top choice of all its applicants as long as $u_j$ does
      not reject them. Since $u_j$ already has $n_j$ applicants, any new
      applicants with a lower rank will automatically be rejected, so the rank
      of $u_j$'s least favorite applicant it does not reject can only stay the
      same or increase. $\blacksquare$
    \end{solution}

    \part Show there does not exist $s_i$, $s_j$, $u_k$ and $u_l$ where $s_i$
          is assigned to $u_k$, $s_j$ is assigned to $u_l$, $s_j$ prefers $u_k$
          to $u_l$ and $u_k$ prefers $s_j$ to $s_i$.
    \begin{solution}
      Proof by contradiction. Assume the above to be false. Since $s_j$ prefers
      $u_k$ to $u_l$ they applied to $u_k$ first. However, $u_j$ is assigned to
      $u_l$, which means they must have been rejected by $u_k$. $s_i$ was not
      rejected by $u_k$ even though they have a lower rank than $s_j$. This
      contradicts the previous proof. $\blacksquare$
    \end{solution}

    \part Show that each student is assigned to its \textit{optimal} university.
    \begin{solution}
      Proof by contradiction. If the assignment by the algorithm is not
      student-optimal, there must be first student (in time), $s_i$, that is
      rejected from its optimal university $u_k$. On this day there must be
      student, $s_j$, that is preferred by $u_k$. Since $u_k$ is in $s_i$'s
      realm of possibility, there is an assignment $M$ that assigns $s_i$ to
      $u_k$. Let $u_l$ be that university that is assigned to $s_j$ in $M$. If
      $s_j$ prefers $u_l$ to $u_k$, then they must have been rejected by $u_l$
      before applying to $u_k$. But this is not possible, since $u_i$ was the
      first student to be rejected by a university in their realm of
      possibility. Thus $s_j$ must prefer $u_k$ to $u_l$. But then $s_j$ and
      $u_k$ prefer each other over their assignments in $M$ (they are a "rouge
      couple"). This is a contradiction since $M$ must not have "rouge couples".
      This means our assumption is wrong and there is no student rejected by
      their optimal university, thus the algorithm assigns every student to
      the optimal university. $\blacksquare$
    \end{solution}
  \end{parts}

\end{questions}
\end{document}
