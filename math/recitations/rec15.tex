\documentclass[../main.tex]{subfiles}

\firstpageheader{6.042}{Recitation Problems 15}{Page \thepage\ of \numpages}
\runningheader{6.042}{Recitation Problems 15}{Page \thepage\ of \numpages}

\begin{document}

\begin{questions}

  \question The Tao of BOOKKEEPER
  \begin{parts}

    \part In how many ways can you arrange the letters in the word \textit{POKE}?
    \begin{solution}
      $4!$ ways
    \end{solution}

    \part In how many ways can you arrange the letters in the word \textit{$BO_1O_2K$}?
    \begin{solution}
      Also $4!$ ways, because $O_1$ and $O_2$ are distinct letters.
    \end{solution}

    \part Map arrangements of the letters in \textit{BOOK} to arrangements of the letters in \textit{$BO_1O_2K$} by removing the subscripts.
    \begin{solution}
      \begin{align*}
        BO_1O_2K &\rightarrow BOOK \\
        BO_2O_1K &\rightarrow BOOK \\
        O_1BO_2K &\rightarrow OBOK \\
        O_2BO_1K &\rightarrow OBOK \\
        KBO_1O_2 &\rightarrow KOBO \\
        KBO_2O_1 &\rightarrow KOBO \\
                 &...
      \end{align*}
    \end{solution}

    \part What kind of mapping is this?
    \begin{solution}
      A 2-to-1 mapping.
    \end{solution}

    \part In light of the division rule, how many arrangements are there of \textit{BOOK}?
    \begin{solution}
      Since there are $4!$ arrangements of \textit{$BO_1O_2K$} and the mapping is 2-to-1, we have $\frac{4!}{2}$ arrangements of \textit{BOOK}.
    \end{solution}

    \part How many arrangements are there of the letters in \textit{$KE_1E_2PE_3R$}?
    \begin{solution}
      $6!$
    \end{solution}

    \part (skipped)

    \part What kind of mapping is this?
    \begin{solution}
      A 3-to-1 mapping.
    \end{solution}

    \pagebreak
    \part So how many arrangements are there of the letters in \textit{KEEPER}?
    \begin{solution}
      $\frac{6!}{3!} = 120$
    \end{solution}

    \part How many arrangements of \textit{$BO_1O_2K_1K_2E_1E_2PE_3R$} are there?
    \begin{solution}
      $10!$
    \end{solution}
  
    \part How many arrangements of \textit{$BOOK_1K_2E_1E_2PE_3R$} are there?
    \begin{solution}
      $\frac{10!}{2!}$
    \end{solution}
  
    \part How many arrangements of \textit{$BOOKKE_1E_2PE_3R$} are there?
    \begin{solution}
      $\frac{10!}{2! \cdot 2!}$
    \end{solution}
  
    \part How many arrangements of \textit{$BOOKKEEPER$} are there?
    \begin{solution}
      $\frac{10!}{2! \cdot 2! \cdot 3!} = 151200$
    \end{solution}

    \part How many arrangements of \textit{VOODOODOLL} are there?
    \begin{solution}
      $\frac{10!}{5! \cdot 2! \cdot 2!} = 7560$
    \end{solution}

    \part How many $n$-bit sequences contain $k$ zeroes and $n-k$ ones?
    \begin{solution}
      $\frac{n!}{k! \cdot (n-k)!}$
    \end{solution}

    This quantity is denoted $\binom{n}{k}$, and read "$n$ choose $k$".
  
  \end{parts}

  \question Pigeonhole Principle
  \begin{parts}

    \part In a room of $500$ people, there exist two who share a birthday.
    \begin{solution}
      There are $366$ pigeonholes, one for every possible birthday.
      There are $500$ pigeons (people).
      Since $500 > 366$, by the pigeonhole principle, at least two people must share a birthday.
    \end{solution}

    \part Suppose that each of the $115$ students in $6.042$ sums the nine digits of their ID number. Must two people arrive at the same sum?
    \begin{solution}
      There are $115$ pigeons (students). The sum must lie between $0$ and $9*9 = 81$, resulting in $82$ pigeonholes. Thus, at least two people must arrive at the same sum.
    \end{solution}

    \part In every set of $100$ integers, there exist two whose difference is a multiple of $37$.
    \begin{solution}
      The pigeons are the $100$ integers. Map each integer to the remainder of dividing it by $37$. Thus there are $37$ pigeonholes ($0$ to $36$). By the pigeonhole principle we know that two numbers must map to the same remainder. Call them $a$ and $b$. Now, since they have the same remainder after dividing by $37$, $a-b$ must be a multiple of $37$.
    \end{solution}
  
  \end{parts}

  \question More Counting Problems
  \begin{parts}

    \part In how many ways can $k$ elements be chosen from an $n$-element set $\{x_1, x_2, ..., x_n\}$?
    \begin{solution}
      There is a bijection from $n$-bit sequences with $k$ ones and $n-k$ zeroes. The sequence $(b_1, b_2, ..., b_n)$ maps to the subset that contains $x_i$ iff $b_i = 1$. Therefore, the number of subsets is $\binom{n}{k}$.
    \end{solution}

    \part How many different ways are there to select a dozen donuts if five varieties are available?
    \begin{solution}
      There is a bijection from $16$-bit sequences with $4$ ones. The sequences are constructed as follows:

      $$
      (<v_1 zeroes>, 1, <v_2 zeroes>, 1, <v_3 zeroes>, 1, <v_4 zeroes>, 1, <v_5 zeroes>)
      $$

      where $v_i$ is the number of donuts chosen from variety $i$.

      This leads to the conclusion that there are $\binom{16}{4}$ valid selections.
    \end{solution}

    \part An independent living group is hosting eight pre-frosh, affectionately known as $P_1,...,P_8$ by the permanent residents. Each pre-frosh is assigned a task: $2$ must wash pots, $2$ must clean the kitchen, $1$ must clean the bathrooms, $1$ must clean the common area, and $2$ must serve dinner. In how many ways can $P_1,...,P_8$ be put to productive use?
    \begin{solution}
      $$
      \frac{8!}{2! \cdot 2! \cdot 2!}
      $$
    \end{solution}

    \part Suppose that two identical $52$-card decks are mixed together. In how many ways can the cards in this double-sized deck be arranged?
    \begin{solution}
      The new set contains $2*52 = 104$ cards and has $2$ of each card:
      $$
      \frac{104!}{(2!)^{52}}
      $$
    \end{solution}
  
  \end{parts}

  \pagebreak
  \question Fun with Phonology: Hawaiian
  \begin{parts}

    \part How many different words are there with exactly $4$ phonemes?
    \begin{solution}
      The possible distributions of vowels and consonants are:

      \begin{center}
        VVVV, CVVV, VCVV, VVCV, CVCV
      \end{center}

      We can use the product rule to calculate the size of each set, and since all the sets are disjoint, we can add the sizes together (sum rule):

      $$
      (25^4) + 3 * (8 * 25^3) + (8^2 * 25^2) = 805625
      $$
    \end{solution}

    \part Now for the general case. Let $A$ be the set of all $n$-phoneme words, and let $A_k$ be the set of all $n$-phoneme words with $k$ consonants. Express $\lvert A \rvert$ in terms of $\lvert A_k \rvert$ for all possible $k$.
    \begin{solution}
      There can be anywhere from $0$ to $n/2$ consonants (we assume $n$ to be even). So:

      $$
      \lvert A \rvert = \sum_{k=0}^{n/2}{\lvert A_k \rvert}
      $$
    \end{solution}

    \part Now let's find $\lvert A_k \rvert$ for an arbitrary $k$. For simplicity's sake, assume Hawaiian has only one constant and one vowel. Find a bijection between $A_k$ and a set of arbitrary sequences of $0$ and $1$ of length $p$. What is $p$?
    \begin{solution}
      Since each consonant has to be followed by a vowel, we can pair each of the $k$ consonants with a vowel, producing $k$ such pairs. These pairs are represented by zeroes in the bit sequence, while the remaining vowels are represented by ones. There are $n-2k$ remaining vowels and the length of the bit sequence $p = k + n - 2k = n - k$.
    \end{solution}

    \part Using this bijection, compute $\lvert A_k \rvert$.
    \begin{solution}
      $\lvert A_k \rvert = \binom{n-k}{n-2k} = \binom{n-k}{k}$.
    \end{solution}

    \part How would you change your expression for $\lvert A_k \rvert$ to allow for $8$ consonants and $25$ vowels, not just one of each?
    \begin{solution}
      $\lvert A_k \rvert = \binom{n-k}{k} \cdot 8^k \cdot 25^{n-k}$.
    \end{solution}

    \part How many $n$-phoneme words are there in Hawaiian?
    \begin{solution}
      $$
      \lvert A \rvert = \sum_{k=0}^{n/2}{\binom{n-k}{k} \cdot 8^k \cdot 25^{n-k}}
      $$
    \end{solution}
  
  \end{parts}

\end{questions}
\end{document}
