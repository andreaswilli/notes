\documentclass[../main.tex]{subfiles}

\firstpageheader{6.042}{Recitation Problems 17}{Page \thepage\ of \numpages}
\runningheader{6.042}{Recitation Problems 17}{Page \thepage\ of \numpages}

\begin{document}

\begin{questions}

  \question The Four-Door Deal
  \begin{parts}

    \part What is the probability of winning with the "stay" strategy?
    \begin{solution}
      It is the same as the probability of picking the winning door in the first phase, so $\frac{1}{4}$.
    \end{solution}

    \part What is the probability of winning with the "switch" strategy?
    \begin{solution}
      For the "switch" strategy to work, we can't pick the winning door in the first phase, so the first part is $\frac{3}{4}$. After a losing door is revealed, there are two doors remaining, one of which is the winning door. Thus, in this phase the probabi
      lity of picking to winning door is $\frac{1}{2}$. So in total:
      $$
      \frac{3}{4} \cdot \frac{1}{2} = \frac{3}{8}
      $$
    \end{solution}
  
  \end{parts}

  \question Earliest Door
  \begin{solution}
    Same as before, if the initial guess is the winning door, we are doomed. So the first step has probability $\frac{3}{4}$.

    Now, there are two cases: With a probability of $\frac{2}{3}$, the first available door is revealed, leaving a $\frac{1}{2}$ probability of winning. With a probability of $\frac{1}{3}$, the first available door is skipped, and we know it must be the winning door.

    In total, we have:
    $$
    \frac{3}{4} \cdot \left( \frac{2}{3} \cdot \frac{1}{2} + \frac{1}{3} \cdot 1 \right) = \frac{3}{4} \cdot \left( \frac{1}{3} + \frac{1}{3} \right) = \frac{1}{2}
    $$
  \end{solution}

  \question The 3 doors version revisited
  \begin{parts}

    \part Carol picks the smallest door
    \begin{solution}
      The probability is still $\frac{2}{3}$, since the only time Carol has a choice, both remaining doors are loosing anyway, so her strategy does not make a difference.
    \end{solution}

    \part Carol picks the smallest door with probability $p$
    \begin{solution}
      Still $\frac{2}{3}$, same reasoning as above.
    \end{solution}

    \part Optimal Strategy
    \begin{solution}
      (skipped)
    \end{solution}
  
  \end{parts}

\end{questions}
\end{document}
