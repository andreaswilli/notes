\documentclass[../main.tex]{subfiles}

\firstpageheader{6.042}{Recitation Problems 19}{Page \thepage\ of \numpages}
\runningheader{6.042}{Recitation Problems 19}{Page \thepage\ of \numpages}

\begin{document}

\begin{questions}

  \question Bayes' Rule
  \begin{parts}

    \part Prove Bayes' Rule
    \begin{solution}
      \begin{align*}
          &Pr\{A | B\} \cdot Pr\{B\} \\
        = &Pr\{A \cap B\} \\
        = &Pr\{B \cap A\} \\
        = &Pr\{B | A\} \cdot Pr\{A\}
      \end{align*}
    \end{solution}

    \part Weatherman
    \begin{solution}
      Let $R$ be the event that it rains, and let $U$ be the event that the weatherman carries an umbrella.

      By Bayes' Rule, we have:
      \begin{align*}
        Pr\{U | R\} \cdot Pr\{R\} &= Pr\{R | U\} \cdot Pr\{U\} \\
        0.8 \cdot 0.3 &= Pr\{R | U\} \cdot 0.4 \\
        Pr\{R | U\} &= \frac{0.8 \cdot 0.3}{0.4} = 0.6 \\
      \end{align*}
    \end{solution}
  
  \end{parts}

  \question DNA Profiles
  \begin{parts}

    \part What is the probability that two people's DNS profile would match?
    \begin{solution}
      We use the birthday problem with number of days $N$ and number of people $m$.

      $$
      N = 20000000000
      $$
      $$
      m = 300000000
      $$
      So the probability that two people's DNS profile would match is:
      $$
      1 - e^{-300000000^2 / (2 \cdot 20000000000)} = 1 - e^{-2250000} \approx 1
      $$
    \end{solution}

    \part Trial
    \begin{solution}
      The above argument shows that there is likely \textit{a} collision. However, if there is only one person with this specific DNA in the database, it's certain this is the person we're looking for.
    \end{solution}
  
  \end{parts}

  \pagebreak
  \question The Immortals
  \begin{parts}

    \part What does your intuition say?
    \begin{solution}
      Intuition cannot be trusted when it comes to probability.
    \end{solution}

    \part What is the probability that the experiment succeeds as a function of $p$ and $n$?
    \begin{solution}
      The event of success $S$ consists of $n$ outcomes where $O_i$ the the outcome where Immortal at position $i$ gets heads while everyone else gets tails. $Pr\{O_i\} = p(1-p)^{n-1}$.

      So $Pr\{S\} = np(1-p)^{n-1}$.
    \end{solution}

    \part How could $p$ be chosen to maximize the probability of $S$?
    \begin{solution}
      
      Let $f(p) = np(1-p)^{n-1}$. To find the maximum we first take the derivative:
      \begin{align*}
        f'(p) &= n((1-p)^{n-1} + p(1-n)(1-p)^{n-2}) \\
              &= n(1-p)^{n-1} + (np-n^2p)(1-p)^{n-2} \\
              &= (n-np)(1-p)^{n-2} + (np-n^2p)(1-p)^{n-2} \\
              &= (n-np + np-n^2p)(1-p)^{n-2} \\
              &= -n(np-1)(1-p)^{n-2} \\
      \end{align*}
      Now we set $f'(p) = 0$ to find the maximum:
      \begin{align*}
        0 &= -n(np-1)(1-p)^{n-2} \\
        p &= \frac{1}{n} \text{ or } 1
      \end{align*}

      But $p=1$ means that the experiment is not successful, so the optimal value for $p$ is $\frac{1}{n}$.
    \end{solution}

    \part What is the probability of success if $p$ is chosen this way?
    \begin{solution}
      With $p = \frac{1}{n}$:
      $$
      Pr\{S\} = \left(1 - \frac{1}{n}\right)^{n-1}
      $$

      When $n$ grows large:
      $$
      \lim_{n \to \infty} \left(1 - \frac{1}{n}\right)^{n-1} = \frac{1}{e}
      $$
      This means the probability of success does not tend to zero if we choose $p$ correctly.
    \end{solution}
  
  \end{parts}

\end{questions}
\end{document}
