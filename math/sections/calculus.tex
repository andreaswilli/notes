\documentclass[../main.tex]{subfiles}
\begin{document}

\section{Calculus}

\subsection{Differentiation}

\subsubsection{Notation}

The following notations are equivalent:

\begin{itemize}
  \item Newton's notation: $f'(x) = y'$
  \item Leibniz' notation: $\frac{df}{dx} = \frac{dy}{dx} = \frac{d}{dx}f = \frac{d}{dx}y$
  \item Other (not sure what it is called): $Df(x) = Dy$
\end{itemize}

Similarly, for derivatives of higher order:

\begin{itemize}
  \item $f''(x) = f^{(2)}(x) = y'' = \frac{d^2y}{dx^2} = \frac{d^2}{dx^2}f(x) = D^2y$
  \item $f'''(x) = f^{(3)}(x) = y''' = \frac{d^3y}{dx^3} = \frac{d^3}{dx^3}f(x) = D^3y$
  \item and so on
\end{itemize}

\subsubsection{Derivative of Polynomials}

$$
  \frac{d}{dx}x^n = nx^{n-1}
$$

\subsubsection{Rules}

\paragraph{Constant Coefficient}

$$
  (cu)' = cu'
$$

"Derivative of a constant times a function is equal to the constant times the derivative of the function."

\paragraph{Sum of Functions}

$$
  (u + v)' = u' + v'
$$

"Derivative of the sum of two functions is the sum of the individual derivatives."

\paragraph{Difference of Functions}

$$
  (u - v)' = u' - v'
$$

"Derivative of the difference of two functions is the difference of the individual derivatives."

\paragraph{Product of Functions}

$$
  (uv)' = u'v + uv'
$$

\paragraph{Quotient of Functions}

$$
  (\frac{u}{v})' = \frac{u'v - uv'}{v^2}
$$

\paragraph{Composite Functions}

$$
  \frac{d}{dx}f(g(x))=f'(g(x))\times g'(x)
$$

\paragraph{Chain Rule}

Let $y=f(u)$ where $u=g(x)$.

$$
  \frac{dy}{dx}=\frac{dy}{du}\times \frac{du}{dx}
$$

\paragraph{Power Rule}

Let $u$ be any differentiable function.

$$
  (u^n)' = nu^{n-1}\times u'
$$

\paragraph{Trigonometric Functions}

$$
  (\sin{x})'=\cos{x}
$$

$$
  (\cos{x})'=-\sin{x}
$$

$$
  (\tan{x})'=-\sec^2{x}
$$

$$
  (\sin^{-1}{x})'=\frac{1}{\sqrt{1-x^2}}
$$

$$
  (\tan^{-1}{x})'=\frac{1}{1+x^2}
$$

\paragraph{Exponentials and Logarithms}

$$
  (\ln{x})'=\frac{1}{x}
$$

$$
  (a^x)'=\ln{a}\times a^x
$$

since $\ln{e}=1$:

$$
  (e^x)'=e^x
$$

\paragraph{Derivatives of Higher Order}

Given $f(x)=x^n$ and $n>0$ and $k>0$:

$$
  f^{(k)}(x)=
  \begin{cases}
    nx^{n-1} & \text{if}\ k=1\\
    \frac{n!}{(n-k)!}x^{n-k} & \text{if}\ 1<k<n\\
    n! & \text{if}\ k=n\\
    0 & \text{if}\ k > n
  \end{cases}
$$

Note: Values for $k=1$ and $k=n$ are obtained by pluggin in $1$ resp.
$n$ in $\frac{n!}{(n-k)!}x^{n-k}$ and simplifying.

Given $f(x)=\frac{1}{x^n}=x^{-n}$ and $n>0$ and $k>0$:

$$
f^{(k)}(x)=(-1)^kk!x^{-(k+n)}
$$

\end{document}
