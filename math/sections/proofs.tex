\documentclass[../main.tex]{subfiles}
\begin{document}

\section{Proofs}

Link to MIT OCW: \href{https://ocw.mit.edu/courses/6-042j-mathematics-for-computer-science-fall-2010/}{6.042 -- Mathematics for Computer Science (2010)}

\subsection{Terminology}

\begin{tabular}{ p{2.5cm} p{13cm} }
  \textbf{Proposition} & a statement that is either true or false \\
  \textbf{Predicate}  & a proposition whose truth depends on one or more
  variables e.g. "n is a perfect square" depends on the value of n \\
  \textbf{Axiom} & a proposition that is universially accepted as true, base
  for further proofs \\
  \textbf{Theorem} & an important true proposition \\
  \textbf{Lemma} & a preliminary proposition useful for proving later proposition \\
  \textbf{Corollary} & a proposition that follows in just a few logical steps
  from a theorem \\
\end{tabular}

\subsection{Notation}

\begin{tabular}{ p{2cm} p{8cm} p{4cm} }
  \textbf{Notation} & \textbf{Read as} & \textbf{Example} \\
  $\forall$ & "for all" & $\forall{n}\in\mathbb{N}.\ f(n)$ is prime. \\
  $\exists$ & "there exists" & $\exists{n}\in\mathbb{N}.\ n > 10.$ \\
  $\frac{\text{antecedents}}{\text{consequent}}$ & "if antecedents are proved, then the consequent (or conclusion) is also proved" & $\frac{P,\ P \implies Q}{Q}$ \\
\end{tabular}

\subsection{Propositions}

\subsubsection{Validity}

A proposition is \textit{valid} if it is always true (for any possible assignment of true and false to individual propositional variables).

Example:

$$
P \lor \neg P
$$

Both possible assignments result in the proposition being true, thus it is valid.

On the other hand:

$$
P \lor Q
$$

This proposition is false if both $P$ and $Q$ are false, thus it is not valid.

\subsubsection{Satisfiability}

A proposition is \textit{satisfiable} if there is an assignment of truth values to its individual propositional variables, such that the proposition is true.

Example:

$$
P \lor Q
$$

This proposition is true if either $P$ or $Q$ is true, thus it is satisfiable.

On the other hand:

$$
P \land \neg P
$$

There exists no assignment such that the proposition is true, thus it is not satisfiable.

\subsection{Logical Deductions (or Inference Rules)}

This inference rule is called \textit{modus ponens}:

$$
\frac{P,\ P \implies Q}{Q}
$$

There are also other inference rules:

$$
\frac
{P \implies Q,\ Q \implies R}
{P \implies R}
$$

$$
\frac{\neg P \implies \neg Q}{Q \implies P}
$$

\subsection{Patterns of Proof}

Proofs typically start with the word "Proof" and end with either "QED" or $\blacksquare$.

\subsubsection{Proving an Implication}
\label{sec:prove_implication}

\paragraph{Method 1}

In order to prove that this is a pretty long text and it will take more than one line. jk, maybe now it will be long enough:

\begin{enumerate}
  \item Write, "Assume $P$."
  \item Show that $Q$ logically follows.
\end{enumerate}

\paragraph{Method 2: Contrapositive}
The implication $P \implies Q$ is equivalent to its \textit{contrapositive} $\neg Q \implies \neg P$.

\begin{enumerate}
  \item Write, "We prove the contrapositive:" and then state the contrapositive.
  \item Proceed as in method 1.
\end{enumerate}

\subsubsection{Proving an "If and Only If"}

\paragraph{Method 1}

$P \iff Q$ is equivalent to the two statements $P \implies Q$ and $Q \implies P$.

\begin{enumerate}
  \item Write, "We prove $P \implies Q$ and vice-versa."
  \item Write, "First, we show $P \implies Q$." Do this by one of the methods in Section~\ref{sec:prove_implication}.
  \item Write, "Now, we show $Q \implies P$." Again, do this by one of the methods in Section~\ref{sec:prove_implication}.
\end{enumerate}

\paragraph{Method 2}

\begin{enumerate}
  \item Write, "We construct a chain of if-and-only-if implications."
  \item Prove $P$ is equivalent to a second statement which is equivalent to a third statement and so on until you reach $Q$.
\end{enumerate}

\subsubsection{Proof by Cases}

A complicated proof can be divided into multiple cases in order to prove each case separately. It's important to also show that the proof covers all the cases.

\begin{enumerate}
  \item Write, "This proof is by case analysis."
  \item Prove each case individually (split further into subcases if necessary).
  \item Show that all the cases are covered and conclude that the initial proposition holds if all the cases hold.
\end{enumerate}

\subsubsection{Proof by Contradiction}

Also called \textit{indirect proof}. This approach is always viable, however, direct proofs are generally preferable when they are available.

In order to prove a proposition $P$ by contradiction:

\begin{enumerate}
  \item Write, "We use proof by contradiction."
  \item Write, "Suppose $P$ is false."
  \item Deduce something known to be false (a logical contradiction).
  \item Write, "This is a contradiction. Therefore, $P$ must be true."
\end{enumerate}

\end{document}
