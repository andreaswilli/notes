\documentclass[../main.tex]{subfiles}
\begin{document}

\section{Sets}

\subsection{Common Sets}

The following are some commonly used sets:

\begin{tabular}{ p{1.5cm} p{9cm} p{4cm} }
  \textbf{Symbol} & \textbf{Set} & \textbf{Example Elements} \\
  $\emptyset$ & Empty set. & none \\
  $\mathbb{N}$ & Set of natural numbers, i.e. non-negative integers. & 69, 420 \\
  $\mathbb{Z}$ & Set of all integer numbers. & -3, 0, 42  \\
  $\mathbb{Q}$ & Set of rational numbers (can be represented by the ratio $\frac{a}{b}$ of two integers). & $\frac{1}{2}$, -$\frac{5}{3}$ \\
  $\mathbb{R}$ & Set of real numbers. & $\pi$, $e$, -9, $\sqrt{2}$ \\
  $\mathbb{C}$ & Set of complex numbers. & $i$, $\frac{19}{2}$, $\sqrt{2}-2i$ \\
\end{tabular}

Additionally, the sets can have a superscript $^+$ or $^-$ to limit the set to its positive resp. negative elements. E.g. $\mathbb{Z}^-$ is the set of all negative integers.

\end{document}
